% Options for packages loaded elsewhere
\PassOptionsToPackage{unicode}{hyperref}
\PassOptionsToPackage{hyphens}{url}
%
\documentclass[
]{book}
\usepackage{amsmath,amssymb}
\usepackage{iftex}
\ifPDFTeX
  \usepackage[T1]{fontenc}
  \usepackage[utf8]{inputenc}
  \usepackage{textcomp} % provide euro and other symbols
\else % if luatex or xetex
  \usepackage{unicode-math} % this also loads fontspec
  \defaultfontfeatures{Scale=MatchLowercase}
  \defaultfontfeatures[\rmfamily]{Ligatures=TeX,Scale=1}
\fi
\usepackage{lmodern}
\ifPDFTeX\else
  % xetex/luatex font selection
\fi
% Use upquote if available, for straight quotes in verbatim environments
\IfFileExists{upquote.sty}{\usepackage{upquote}}{}
\IfFileExists{microtype.sty}{% use microtype if available
  \usepackage[]{microtype}
  \UseMicrotypeSet[protrusion]{basicmath} % disable protrusion for tt fonts
}{}
\makeatletter
\@ifundefined{KOMAClassName}{% if non-KOMA class
  \IfFileExists{parskip.sty}{%
    \usepackage{parskip}
  }{% else
    \setlength{\parindent}{0pt}
    \setlength{\parskip}{6pt plus 2pt minus 1pt}}
}{% if KOMA class
  \KOMAoptions{parskip=half}}
\makeatother
\usepackage{xcolor}
\usepackage{color}
\usepackage{fancyvrb}
\newcommand{\VerbBar}{|}
\newcommand{\VERB}{\Verb[commandchars=\\\{\}]}
\DefineVerbatimEnvironment{Highlighting}{Verbatim}{commandchars=\\\{\}}
% Add ',fontsize=\small' for more characters per line
\usepackage{framed}
\definecolor{shadecolor}{RGB}{248,248,248}
\newenvironment{Shaded}{\begin{snugshade}}{\end{snugshade}}
\newcommand{\AlertTok}[1]{\textcolor[rgb]{0.94,0.16,0.16}{#1}}
\newcommand{\AnnotationTok}[1]{\textcolor[rgb]{0.56,0.35,0.01}{\textbf{\textit{#1}}}}
\newcommand{\AttributeTok}[1]{\textcolor[rgb]{0.13,0.29,0.53}{#1}}
\newcommand{\BaseNTok}[1]{\textcolor[rgb]{0.00,0.00,0.81}{#1}}
\newcommand{\BuiltInTok}[1]{#1}
\newcommand{\CharTok}[1]{\textcolor[rgb]{0.31,0.60,0.02}{#1}}
\newcommand{\CommentTok}[1]{\textcolor[rgb]{0.56,0.35,0.01}{\textit{#1}}}
\newcommand{\CommentVarTok}[1]{\textcolor[rgb]{0.56,0.35,0.01}{\textbf{\textit{#1}}}}
\newcommand{\ConstantTok}[1]{\textcolor[rgb]{0.56,0.35,0.01}{#1}}
\newcommand{\ControlFlowTok}[1]{\textcolor[rgb]{0.13,0.29,0.53}{\textbf{#1}}}
\newcommand{\DataTypeTok}[1]{\textcolor[rgb]{0.13,0.29,0.53}{#1}}
\newcommand{\DecValTok}[1]{\textcolor[rgb]{0.00,0.00,0.81}{#1}}
\newcommand{\DocumentationTok}[1]{\textcolor[rgb]{0.56,0.35,0.01}{\textbf{\textit{#1}}}}
\newcommand{\ErrorTok}[1]{\textcolor[rgb]{0.64,0.00,0.00}{\textbf{#1}}}
\newcommand{\ExtensionTok}[1]{#1}
\newcommand{\FloatTok}[1]{\textcolor[rgb]{0.00,0.00,0.81}{#1}}
\newcommand{\FunctionTok}[1]{\textcolor[rgb]{0.13,0.29,0.53}{\textbf{#1}}}
\newcommand{\ImportTok}[1]{#1}
\newcommand{\InformationTok}[1]{\textcolor[rgb]{0.56,0.35,0.01}{\textbf{\textit{#1}}}}
\newcommand{\KeywordTok}[1]{\textcolor[rgb]{0.13,0.29,0.53}{\textbf{#1}}}
\newcommand{\NormalTok}[1]{#1}
\newcommand{\OperatorTok}[1]{\textcolor[rgb]{0.81,0.36,0.00}{\textbf{#1}}}
\newcommand{\OtherTok}[1]{\textcolor[rgb]{0.56,0.35,0.01}{#1}}
\newcommand{\PreprocessorTok}[1]{\textcolor[rgb]{0.56,0.35,0.01}{\textit{#1}}}
\newcommand{\RegionMarkerTok}[1]{#1}
\newcommand{\SpecialCharTok}[1]{\textcolor[rgb]{0.81,0.36,0.00}{\textbf{#1}}}
\newcommand{\SpecialStringTok}[1]{\textcolor[rgb]{0.31,0.60,0.02}{#1}}
\newcommand{\StringTok}[1]{\textcolor[rgb]{0.31,0.60,0.02}{#1}}
\newcommand{\VariableTok}[1]{\textcolor[rgb]{0.00,0.00,0.00}{#1}}
\newcommand{\VerbatimStringTok}[1]{\textcolor[rgb]{0.31,0.60,0.02}{#1}}
\newcommand{\WarningTok}[1]{\textcolor[rgb]{0.56,0.35,0.01}{\textbf{\textit{#1}}}}
\usepackage{longtable,booktabs,array}
\usepackage{calc} % for calculating minipage widths
% Correct order of tables after \paragraph or \subparagraph
\usepackage{etoolbox}
\makeatletter
\patchcmd\longtable{\par}{\if@noskipsec\mbox{}\fi\par}{}{}
\makeatother
% Allow footnotes in longtable head/foot
\IfFileExists{footnotehyper.sty}{\usepackage{footnotehyper}}{\usepackage{footnote}}
\makesavenoteenv{longtable}
\usepackage{graphicx}
\makeatletter
\def\maxwidth{\ifdim\Gin@nat@width>\linewidth\linewidth\else\Gin@nat@width\fi}
\def\maxheight{\ifdim\Gin@nat@height>\textheight\textheight\else\Gin@nat@height\fi}
\makeatother
% Scale images if necessary, so that they will not overflow the page
% margins by default, and it is still possible to overwrite the defaults
% using explicit options in \includegraphics[width, height, ...]{}
\setkeys{Gin}{width=\maxwidth,height=\maxheight,keepaspectratio}
% Set default figure placement to htbp
\makeatletter
\def\fps@figure{htbp}
\makeatother
\setlength{\emergencystretch}{3em} % prevent overfull lines
\providecommand{\tightlist}{%
  \setlength{\itemsep}{0pt}\setlength{\parskip}{0pt}}
\setcounter{secnumdepth}{5}
\usepackage{booktabs}
\ifLuaTeX
  \usepackage{selnolig}  % disable illegal ligatures
\fi
\usepackage[]{natbib}
\bibliographystyle{plainnat}
\IfFileExists{bookmark.sty}{\usepackage{bookmark}}{\usepackage{hyperref}}
\IfFileExists{xurl.sty}{\usepackage{xurl}}{} % add URL line breaks if available
\urlstyle{same}
\hypersetup{
  pdftitle={Theory-driven analysis of ecological data: a practical handbook},
  pdfauthor={us},
  hidelinks,
  pdfcreator={LaTeX via pandoc}}

\title{Theory-driven analysis of ecological data: a practical handbook}
\author{us}
\date{2023-10-10}

\usepackage{amsthm}
\newtheorem{theorem}{Theorem}[chapter]
\newtheorem{lemma}{Lemma}[chapter]
\newtheorem{corollary}{Corollary}[chapter]
\newtheorem{proposition}{Proposition}[chapter]
\newtheorem{conjecture}{Conjecture}[chapter]
\theoremstyle{definition}
\newtheorem{definition}{Definition}[chapter]
\theoremstyle{definition}
\newtheorem{example}{Example}[chapter]
\theoremstyle{definition}
\newtheorem{exercise}{Exercise}[chapter]
\theoremstyle{definition}
\newtheorem{hypothesis}{Hypothesis}[chapter]
\theoremstyle{remark}
\newtheorem*{remark}{Remark}
\newtheorem*{solution}{Solution}
\begin{document}
\maketitle

{
\setcounter{tocdepth}{1}
\tableofcontents
}
\chapter{About Bookdown}\label{about-bookdown}

This is a \emph{sample} book written in \textbf{Markdown}. You can use anything that Pandoc's Markdown supports; for example, a math equation \(a^2 + b^2 = c^2\).

\section{Usage}\label{usage}

Each \textbf{bookdown} chapter is an .Rmd file, and each .Rmd file can contain one (and only one) chapter. A chapter \emph{must} start with a first-level heading: \texttt{\#\ A\ good\ chapter}, and can contain one (and only one) first-level heading.

Use second-level and higher headings within chapters like: \texttt{\#\#\ A\ short\ section} or \texttt{\#\#\#\ An\ even\ shorter\ section}.

The \texttt{index.Rmd} file is required, and is also your first book chapter. It will be the homepage when you render the book.

\section{Render book}\label{render-book}

You can render the HTML version of this example book without changing anything:

\begin{enumerate}
\def\labelenumi{\arabic{enumi}.}
\item
  Find the \textbf{Build} pane in the RStudio IDE, and
\item
  Click on \textbf{Build Book}, then select your output format, or select ``All formats'' if you'd like to use multiple formats from the same book source files.
\end{enumerate}

Or build the book from the R console:

\begin{Shaded}
\begin{Highlighting}[]
\NormalTok{bookdown}\SpecialCharTok{::}\FunctionTok{render\_book}\NormalTok{()}
\end{Highlighting}
\end{Shaded}

To render this example to PDF as a \texttt{bookdown::pdf\_book}, you'll need to install XeLaTeX. You are recommended to install TinyTeX (which includes XeLaTeX): \url{https://yihui.org/tinytex/}.

\section{Preview book}\label{preview-book}

As you work, you may start a local server to live preview this HTML book. This preview will update as you edit the book when you save individual .Rmd files. You can start the server in a work session by using the RStudio add-in ``Preview book'', or from the R console:

\begin{Shaded}
\begin{Highlighting}[]
\NormalTok{bookdown}\SpecialCharTok{::}\FunctionTok{serve\_book}\NormalTok{()}
\end{Highlighting}
\end{Shaded}

\section{Here are some useful things for writing the book using bookdown}\label{here-are-some-useful-things-for-writing-the-book-using-bookdown}

All chapters start with a first-level heading followed by your chapter title, like the line above. There should be only one first-level heading (\texttt{\#}) per .Rmd file.

\subsection{A section}\label{a-section}

All chapter sections start with a second-level (\texttt{\#\#}) or higher heading followed by your section title, like the sections above and below here. You can have as many as you want within a chapter.

\subsubsection*{An unnumbered section}\label{an-unnumbered-section}
\addcontentsline{toc}{subsubsection}{An unnumbered section}

Chapters and sections are numbered by default. To un-number a heading, add a \texttt{\{.unnumbered\}} or the shorter \texttt{\{-\}} at the end of the heading, like in this section.

\section{Cross-references}\label{cross}

Cross-references make it easier for your readers to find and link to elements in your book.

\subsection{Chapters and sub-chapters}\label{chapters-and-sub-chapters}

There are two steps to cross-reference any heading:

\begin{enumerate}
\def\labelenumi{\arabic{enumi}.}
\tightlist
\item
  Label the heading: \texttt{\#\ Hello\ world\ \{\#nice-label\}}.

  \begin{itemize}
  \tightlist
  \item
    Leave the label off if you like the automated heading generated based on your heading title: for example, \texttt{\#\ Hello\ world} = \texttt{\#\ Hello\ world\ \{\#hello-world\}}.
  \item
    To label an un-numbered heading, use: \texttt{\#\ Hello\ world\ \{-\#nice-label\}} or \texttt{\{\#\ Hello\ world\ .unnumbered\}}.
  \end{itemize}
\item
  Next, reference the labeled heading anywhere in the text using \texttt{\textbackslash{}@ref(nice-label)}; for example, please see Chapter \ref{cross}.

  \begin{itemize}
  \tightlist
  \item
    If you prefer text as the link instead of a numbered reference use: \hyperref[cross]{any text you want can go here}.
  \end{itemize}
\end{enumerate}

\subsection{Captioned figures and tables}\label{captioned-figures-and-tables}

Figures and tables \emph{with captions} can also be cross-referenced from elsewhere in your book using \texttt{\textbackslash{}@ref(fig:chunk-label)} and \texttt{\textbackslash{}@ref(tab:chunk-label)}, respectively.

See Figure \ref{fig:nice-fig}.

\begin{Shaded}
\begin{Highlighting}[]
\FunctionTok{par}\NormalTok{(}\AttributeTok{mar =} \FunctionTok{c}\NormalTok{(}\DecValTok{4}\NormalTok{, }\DecValTok{4}\NormalTok{, .}\DecValTok{1}\NormalTok{, .}\DecValTok{1}\NormalTok{))}
\FunctionTok{plot}\NormalTok{(pressure, }\AttributeTok{type =} \StringTok{\textquotesingle{}b\textquotesingle{}}\NormalTok{, }\AttributeTok{pch =} \DecValTok{19}\NormalTok{)}
\end{Highlighting}
\end{Shaded}

\begin{figure}

{\centering \includegraphics[width=0.8\linewidth]{_main_files/figure-latex/nice-fig-1} 

}

\caption{Here is a nice figure!}\label{fig:nice-fig}
\end{figure}

Don't miss Table \ref{tab:nice-tab}.

\begin{Shaded}
\begin{Highlighting}[]
\NormalTok{knitr}\SpecialCharTok{::}\FunctionTok{kable}\NormalTok{(}
  \FunctionTok{head}\NormalTok{(pressure, }\DecValTok{10}\NormalTok{), }\AttributeTok{caption =} \StringTok{\textquotesingle{}Here is a nice table!\textquotesingle{}}\NormalTok{,}
  \AttributeTok{booktabs =} \ConstantTok{TRUE}
\NormalTok{)}
\end{Highlighting}
\end{Shaded}

\begin{table}

\caption{\label{tab:nice-tab}Here is a nice table!}
\centering
\begin{tabular}[t]{rr}
\toprule
temperature & pressure\\
\midrule
0 & 0.0002\\
20 & 0.0012\\
40 & 0.0060\\
60 & 0.0300\\
80 & 0.0900\\
\addlinespace
100 & 0.2700\\
120 & 0.7500\\
140 & 1.8500\\
160 & 4.2000\\
180 & 8.8000\\
\bottomrule
\end{tabular}
\end{table}

\section{Parts}\label{parts}

You can add parts to organize one or more book chapters together. Parts can be inserted at the top of an .Rmd file, before the first-level chapter heading in that same file.

Add a numbered part: \texttt{\#\ (PART)\ Act\ one\ \{-\}} (followed by \texttt{\#\ A\ chapter})

Add an unnumbered part: \texttt{\#\ (PART\textbackslash{}*)\ Act\ one\ \{-\}} (followed by \texttt{\#\ A\ chapter})

Add an appendix as a special kind of un-numbered part: \texttt{\#\ (APPENDIX)\ Other\ stuff\ \{-\}} (followed by \texttt{\#\ A\ chapter}). Chapters in an appendix are prepended with letters instead of numbers.

\section{Footnotes and citations}\label{footnotes-and-citations}

\subsection{Footnotes}\label{footnotes}

Footnotes are put inside the square brackets after a caret \texttt{\^{}{[}{]}}. Like this one \footnote{This is a footnote.}.

\subsection{Citations}\label{citations}

Reference items in your bibliography file(s) using \texttt{@key}.

For example, we are using the \textbf{bookdown} package \citep{R-bookdown} (check out the last code chunk in index.Rmd to see how this citation key was added) in this sample book, which was built on top of R Markdown and \textbf{knitr} \citep{xie2015} (this citation was added manually in an external file book.bib).
Note that the \texttt{.bib} files need to be listed in the index.Rmd with the YAML \texttt{bibliography} key.

The RStudio Visual Markdown Editor can also make it easier to insert citations: \url{https://rstudio.github.io/visual-markdown-editing/\#/citations}

\section{Blocks}\label{blocks}

\subsection{Equations}\label{equations}

Here is an equation.

\begin{equation} 
  f\left(k\right) = \binom{n}{k} p^k\left(1-p\right)^{n-k}
  \label{eq:binom}
\end{equation}

You may refer to using \texttt{\textbackslash{}@ref(eq:binom)}, like see Equation \eqref{eq:binom}.

\subsection{Theorems and proofs}\label{theorems-and-proofs}

Labeled theorems can be referenced in text using \texttt{\textbackslash{}@ref(thm:tri)}, for example, check out this smart theorem \ref{thm:tri}.

\begin{theorem}
\protect\hypertarget{thm:tri}{}\label{thm:tri}For a right triangle, if \(c\) denotes the \emph{length} of the hypotenuse
and \(a\) and \(b\) denote the lengths of the \textbf{other} two sides, we have
\[a^2 + b^2 = c^2\]
\end{theorem}

Read more here \url{https://bookdown.org/yihui/bookdown/markdown-extensions-by-bookdown.html}.

\subsection{Callout blocks}\label{callout-blocks}

The R Markdown Cookbook provides more help on how to use custom blocks to design your own callouts: \url{https://bookdown.org/yihui/rmarkdown-cookbook/custom-blocks.html}

\section{Sharing your book}\label{sharing-your-book}

\subsection{Publishing}\label{publishing}

HTML books can be published online, see: \url{https://bookdown.org/yihui/bookdown/publishing.html}

\subsection{404 pages}\label{pages}

By default, users will be directed to a 404 page if they try to access a webpage that cannot be found. If you'd like to customize your 404 page instead of using the default, you may add either a \texttt{\_404.Rmd} or \texttt{\_404.md} file to your project root and use code and/or Markdown syntax.

\subsection{Metadata for sharing}\label{metadata-for-sharing}

Bookdown HTML books will provide HTML metadata for social sharing on platforms like Twitter, Facebook, and LinkedIn, using information you provide in the \texttt{index.Rmd} YAML. To setup, set the \texttt{url} for your book and the path to your \texttt{cover-image} file. Your book's \texttt{title} and \texttt{description} are also used.

This \texttt{gitbook} uses the same social sharing data across all chapters in your book- all links shared will look the same.

Specify your book's source repository on GitHub using the \texttt{edit} key under the configuration options in the \texttt{\_output.yml} file, which allows users to suggest an edit by linking to a chapter's source file.

Read more about the features of this output format here:

\url{https://pkgs.rstudio.com/bookdown/reference/gitbook.html}

Or use:

\begin{Shaded}
\begin{Highlighting}[]
\NormalTok{?bookdown}\SpecialCharTok{::}\NormalTok{gitbook}
\end{Highlighting}
\end{Shaded}

\chapter{Preambule}\label{preambule}

Who is the textbook for?

\chapter{Background (more fancy title needed)}\label{background-more-fancy-title-needed}

\section{Modelling in ecology}\label{modelling-in-ecology}

\subsection{A quick history}\label{a-quick-history}

\subsection{Specificities of ecology}\label{specificities-of-ecology}

Variability + time*space + local interactions + observations through both experiments and field studies + open systems
Formalisation,simplifications often used in ecology, assumptions, \ldots{}

\subsection{Ecological data, uncertainties, sampling}\label{ecological-data-uncertainties-sampling}

At different organisational scales: individual, population, community, ecosystem and all meta-

\subsection{Mathematical modelling / computational approaches (lots to discuss here from Game of life type stuff/von Neumann to IBMs/ABMs note: Grimm is not universally representative ;-) ) ?}\label{mathematical-modelling-computational-approaches-lots-to-discuss-here-from-game-of-life-type-stuffvon-neumann-to-ibmsabms-note-grimm-is-not-universally-representative--}

\subsection{Appendices: Primer on vectors and matrices (multidimensional data)}\label{appendices-primer-on-vectors-and-matrices-multidimensional-data}

\subsection{Appendices: Primer on analysis, dérivées, Taylor series, minimum/maximum\ldots{}}\label{appendices-primer-on-analysis-duxe9rivuxe9es-taylor-series-minimummaximum}

\subsection{Appendices: Removing dimensions, dimensional analysis}\label{appendices-removing-dimensions-dimensional-analysis}

\chapter{Mathematical (or ``process-based''?) modeling}\label{mathematical-or-process-based-modeling}

Intro pour expliquer la logique de l'agencement des chapitres

\section{Finite numbers of individuals (stochastic models)}\label{finite-numbers-of-individuals-stochastic-models}

\subsection{Stochastic individual-based simulations}\label{stochastic-individual-based-simulations}

\subsection{Primer on probabilities (random variables, discrete and continuous distributions, PGF and MGF, central limit theorem)}\label{primer-on-probabilities-random-variables-discrete-and-continuous-distributions-pgf-and-mgf-central-limit-theorem}

\subsection{Small populations and branching processes}\label{small-populations-and-branching-processes}

\subsection{Markov chains (ergodicity, absorbing states)}\label{markov-chains-ergodicity-absorbing-states}

\subsection{Master equations, moment closure?}\label{master-equations-moment-closure}

\subsection{PDEs, diffusion}\label{pdes-diffusion}

\subsection{Boxes: algo Gillespie + refinements ; generating random numbers according to defined distributions}\label{boxes-algo-gillespie-refinements-generating-random-numbers-according-to-defined-distributions}

\section{Large numbers (dynamical systems; and ways to analyse them)}\label{large-numbers-dynamical-systems-and-ways-to-analyse-them}

\subsection{Primer on analysis (derivatives, attractors, stability, permanence, jacobian, changes of variables\ldots{} + analytical approximations: Taylor series, delta method)}\label{primer-on-analysis-derivatives-attractors-stability-permanence-jacobian-changes-of-variables-analytical-approximations-taylor-series-delta-method}

\subsection{Difference equations (ex: fibonacci)}\label{difference-equations-ex-fibonacci}

\subsection{ODEs (example: malthus and logistic)}\label{odes-example-malthus-and-logistic}

\subsection{Stage-structured / physiologically structured / compartment models (R0, Euler-Lotka, reproductive values, eigenvalue-based growth rate\ldots)}\label{stage-structured-physiologically-structured-compartment-models-r0-euler-lotka-reproductive-values-eigenvalue-based-growth-rate}

\subsection{Boxes: integration schemes (Euler et al.)}\label{boxes-integration-schemes-euler-et-al.}

\section{Reintroducing stochasticity (dynamical systems with noise)}\label{reintroducing-stochasticity-dynamical-systems-with-noise}

\subsection{Environmental stochasticity / demographic stochasticity}\label{environmental-stochasticity-demographic-stochasticity}

\section{Spatial structure}\label{spatial-structure}

\subsection{Implicit space and patches}\label{implicit-space-and-patches}

\subsection{Explicit discrete space (networks and lattices)}\label{explicit-discrete-space-networks-and-lattices}

\subsection{Explicit continuous spaces (PDEs)}\label{explicit-continuous-spaces-pdes}

\subsection{Models of dispersal and foraging (kernels, brownian motion, Lévy walk, optimal foraging\ldots)}\label{models-of-dispersal-and-foraging-kernels-brownian-motion-luxe9vy-walk-optimal-foraging}

\section{Interactions}\label{interactions}

\subsection{Types of interactions and functional responses (in continuous and discrete time)}\label{types-of-interactions-and-functional-responses-in-continuous-and-discrete-time}

Chemical kinetics to invent functional responses

\subsection{Lotka-Volterra}\label{lotka-volterra}

\subsection{Food-web models (niche, cascade, \ldots)}\label{food-web-models-niche-cascade}

\subsection{Network analysis (primer)}\label{network-analysis-primer}

\subsection{Random matrices}\label{random-matrices}

\section{Links between ecology and evolution}\label{links-between-ecology-and-evolution}

\subsection{Historical overview (incl.~common processes cf Vellend etc; globally cf.~for example Huneman 2019; and maybe ELE special issue 2023)}\label{historical-overview-incl.-common-processes-cf-vellend-etc-globally-cf.-for-example-huneman-2019-and-maybe-ele-special-issue-2023}

\subsection{Timescales}\label{timescales}

\subsection{Frameworks (AD, QG, oligomorphic dynamics, IBMs)}\label{frameworks-ad-qg-oligomorphic-dynamics-ibms}

\chapter{Linking process-based models to data - the statistical interface}\label{linking-process-based-models-to-data---the-statistical-interface}

\section{Linking process-based models to data: the statistical interface}\label{linking-process-based-models-to-data-the-statistical-interface}

\subsection{How to confront models and data: from qualitative to quantitative (overview of existing lit)}\label{how-to-confront-models-and-data-from-qualitative-to-quantitative-overview-of-existing-lit}

\subsection{Models on sampling uncertainty, sources of variability (types of error, non-observed states etc?)}\label{models-on-sampling-uncertainty-sources-of-variability-types-of-error-non-observed-states-etc}

\subsection{Learning and validation}\label{learning-and-validation}

\section{Model fitting}\label{model-fitting}

\subsection{Distances between model predictions and data (least square\ldots)}\label{distances-between-model-predictions-and-data-least-square}

\subsection{Likelihood}\label{likelihood}

Bayesian,
MCMC,
ABC

\section{Model selection}\label{model-selection}

\subsection{Model selection and Information criteria, model averaging/Burnham-Anderson stuff}\label{model-selection-and-information-criteria-model-averagingburnham-anderson-stuff}

\subsection{Penalised regression (lasso, ridge, etc.)}\label{penalised-regression-lasso-ridge-etc.}

\subsection{Machine learning approaches (random forest, neural networks, \ldots)}\label{machine-learning-approaches-random-forest-neural-networks}

  \bibliography{book.bib,packages.bib}

\end{document}
